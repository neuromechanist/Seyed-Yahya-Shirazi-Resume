%% If you are using \orcid or academicons
%% icons, make sure you have the academicons
%% option here, and compile with XeLaTeX
%% or LuaLaTeX.
% \documentclass[10pt,a4paper,academicons]{altacv}

%% Use the "normalphoto" option if you want a normal photo instead of cropped to a circle
% \documentclass[10pt,a4paper,normalphoto]{altacv}

\documentclass[10pt,letter,ragged2e]{altacv}

%% AltaCV uses the fontawesome and academicon fonts
%% and packages.
%% See texdoc.net/pkg/fontawecome and http://texdoc.net/pkg/academicons for full list of symbols. You MUST compile with XeLaTeX or LuaLaTeX if you want to use academicons.

% Change the page layout if you need to
\geometry{left=2cm,right=10cm,marginparwidth=6.8cm,marginparsep=1.2cm,top=1.25cm,bottom=1.25cm}
\usepackage{hyperref}
% Change the font if you want to, depending on whether
% you're using pdflatex or xelatex/lualatex
\ifxetexorluatex
  % If using xelatex or lualatex:
  \setmainfont{Carlito}
\else
  % If using pdflatex:
  \usepackage[utf8]{inputenc}
  \usepackage[T1]{fontenc}
  \usepackage[default]{lato}
  \usepackage{outlines}
\fi

% Change the colours if you want to
\definecolor{VividPurple}{HTML}{000000}
\definecolor{SlateGrey}{HTML}{2E2E2E}
\definecolor{LightGrey}{HTML}{2E2E2E}
\colorlet{heading}{VividPurple}
\colorlet{accent}{VividPurple}
\colorlet{emphasis}{SlateGrey}
\colorlet{body}{LightGrey}

% Change the bullets for itemize and rating marker
% for \cvskill if you want to
\renewcommand{\itemmarker}{{\small\textbullet}}
\renewcommand{\ratingmarker}{\faCircle}

%% sample.bib contains your publications
\addbibresource{sample.bib}

\begin{document}
\name{Seyed Yahya Shirazi, Ph.D.}
\tagline{Postdoctoral Associate, New York University}
% Cropped to square from https://en.wikipedia.org/wiki/Marissa_Mayer#/media/File:Marissa_Mayer_May_2014_(cropped).jpg, CC-BY 2.0
%\photo{3.3cm}{profile.jpg}
\personalinfo{%
  % Not all of these are required!
  % You can add your own with \printinfo{symbol}{detail}
  \email{\href{mailto:shirazi@ieee.org}{shirazi@ieee.org}}
  \phone{407-801-0090}
%  \mailaddress{Address, Street, 00000 County}
  % \location{New York, NY}
%  \homepage{marissamayr.tumblr.com/}
\homepage{\href{https://neuromechanist.github.io}{neuromechanist.github.io}}
%  \twitter{\href{https://twitter.com/shirazi_en}{@shirazi\_en}}
\linkedin{\href{https://www.linkedin.com/in/seyedyahya/}{/seyedyahya}}
% \github{\href{https://github.com/neuromechanist}{/neuromechanist}} % I'm just making this up though.
%   \orcid{orcid.org/0000-0000-0000-0000} % Obviously making this up too. If you want to use this field (and also other academicons symbols), add "academicons" option to \documentclass{altacv}
}

%% Make the header extend all the way to the right, if you want.
\begin{fullwidth}
\makecvheader
\end{fullwidth}

%% Depending on your tastes, you may want to make fonts of itemize environments slightly smaller
\AtBeginEnvironment{itemize}{\small}

%% Provide the file name containing the sidebar contents as an optional parameter to \cvsection.
%% You can always just use \marginpar{...} if you do
%% not need to align the top of the contents to any
%% \cvsection title in the "main" bar.
\cvsection[side1]{Experience \small{\tt{(selected)}}}

\cvevent{Postdoctoral Associate}{New York University}{May 2021 -- Present}{New York, NY}
\begin{itemize}
\item Study the cortico-muscular connectivity during stroke rehabilitation using high-density EEG and high-density EMG for an NSF/FDA-funded project. I aim to explore objective biomarkers to evaluate the efficacy of rehabilitation devices.
\smallskip
\item Study muscle network biomarkers of stroke rehabilitation and fatigue. We quantify changes in intermuscular connectivity using network metrics.
\end{itemize}

\divider

\cvevent{Graduate Research Assistant}{University of Central Florida}{Jan. 2017 -- April 2021}{Orlando, FL}
\begin{itemize}
\item Studied young and older adults' responses to mechanical perturbations during a seated exercise. I collected \& analyzed EEG, EMG \& biomechanical data. Analyses included biomechanical behavior, EEG spectrotenporal analysis \& cortico-muscular connectivity.
\item Developed new 3D position recording methods to digitize EEG electrode locations, using a motion capture system abd 3D scanners. I implemented several image-processing \& clustering techniques including iterative closest point (ICP) \& Gaussian mixture model (GMM).

\item Constructed deep neural network models for online classification of EEG signals to find movement intention in a locomotor task prior to movement execution.
\end{itemize}

%\divider

\cvsection{Education}

\cvevent{Doctor of Philosophy}{University of Central Florida | Mechanical Engineering }{Jan. 2017 -- April, 2021}{}
\begin{itemize}
\item Dissertation : Corticomuscular adaptation to mechanical perturbations in a seated locomotor task
% \item GPA: 3.75
\end{itemize}

\divider

\cvevent{Master of Science}{Tehran Polytechnic  | Biomedical Engineering}{Sep. 2011 -- Feb. 2014}{}
\begin{itemize}
\item Thesis : Dynamic Postural Stability Analysis on Standing Normal Subjects \& Transtibial Amputees.
% \smallskip
% \item GPA: 3.76
\end{itemize}

\divider

\cvevent{Bachelor of Science (with Honors)}{Tehran Polytechnic  | Biomedical Engineering}{Sep. 2007 -- Sep. 2011}{}
\begin{itemize}
% \item Bachelor Project: Development of a Stability Analyzer featuring a
% Perturbation Module
% % \smallskip
% \item GPA: 3.70 
\item top 3rd in the department
\end{itemize}

\clearpage


\nocite{*}

\end{document}
