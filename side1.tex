\cvsection{Skills}
\begin{outline}
    \1 Biomedical signal acquisition
        \2[--] {\footnotesize HD-EEG, HD-EMG, EMG, Motion Capture (reflective marker and markerless), Kinectics}
    \1 Signal processing and machine learning
        \2[--] {\footnotesize Matlab: EEGLAB, SP, NN, DL, Image processing}
        \2[--] {\footnotesize Simulink: IoT (Raspberry, Arduino), Control}
        \2[--] {\footnotesize Python: Numpy, Pandas, Scipy, Scikit Learn, MNE}
        \2[--] {\footnotesize GitHub, VSCode, Docker, Ray (futures), Dask}
    \1 Human 3D modeling
        \2[--] {\footnotesize Mimics, XOR}
    \1 Product development
        \2[--] {\footnotesize SolidWorks, Ansys (Structural, CFD, FSI), Visio}
    
\end{outline}

\cvsection{projects \small{\tt{(ECA, selected)}}}
\cvproject{Perturbations on-the-go}
\begin{itemize}
\item Developed a real-time controller to create resistive perturbations during stepping exercise.
\item The controller is scalable to small to small exercise devices using an IoT kit.
% \item Perturbations can be applied at different time windows with variable intensities and durations to selectively engage different response mechanisms of the nervous system.
\end{itemize}
\smallskip
\cvproject{Zombie ant biomechanics using ResNet}
\begin{itemize}
\item Implemented DeepLabCut toolbox (a ResNet network for pose estimation and tracking) to track antennae and limb segments of Carpenter ants before and after infecting with Ophiocordyceps unilateralis (Zombie) fungus. {\footnotesize{link to the video:} {\tt{\href{https://pic.twitter.com/59Qk9fLJHU}{pic.twitter.com/59Qk9fLJHU}}}}
\end{itemize}

\cvsection{Publications \small{\tt{(selected)}}}
\smallskip
\textbf{Shirazi, S. Y.} \& Huang, H. J. \textit{Differential theta-band signatures of the anterior cingulate and motor cortices during seated locomotor perturbations}, IEEE TNSRE, 2021. {\footnotesize{link to the article:} {\tt{\href{https://ieeexplore.ieee.org/document/9347561}{10.1109/tnsre.2021.3057054}}}}
\smallskip
\vspace{1ex}

\textbf{Shirazi, S. Y.} \& Huang, H. J. \textit{More reliable EEG electrode digitizing methods can reduce source estimation uncertainty, but current methods already accurately identify Brodmann areas}, Frontiers in Neuroscience, 2019. {\footnotesize{link to the article:} {\tt{\href{https://www.frontiersin.org/articles/10.3389/fnins.2019.01159/}{10.3389/fnins.2019.01159}}}}

\cvsection{Patent}
\textbf{S.Y.Shirazi}: \textit{Centrifugal Micro-viscometer. A lab-on-a-chip device to assess viscosity of biological fluids}, Iran Patent \#77944, June 2012.
